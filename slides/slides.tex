%%%%%%%%%%%%%%%%%%%%%%%%%%%%%%%%%%%%%%%%%
% Beamer Presentation
% LaTeX Template
% Version 1.0 (10/11/12)
%
% This template has been downloaded from:
% http://www.LaTeXTemplates.com
%
% License:
% CC BY-NC-SA 3.0 (http://creativecommons.org/licenses/by-nc-sa/3.0/)
%
%%%%%%%%%%%%%%%%%%%%%%%%%%%%%%%%%%%%%%%%%

%----------------------------------------------------------------------------------------
%	PACKAGES AND THEMES
%----------------------------------------------------------------------------------------

\documentclass{beamer}

\mode<presentation> {

% The Beamer class comes with a number of default slide themes
% which change the colors and layouts of slides. Below this is a list
% of all the themes, uncomment each in turn to see what they look like.

%\usetheme{default}
%\usetheme{AnnArbor}
%\usetheme{Antibes}
%\usetheme{Bergen}
%\usetheme{Berkeley}
%\usetheme{Berlin}
%\usetheme{Boadilla}
%\usetheme{CambridgeUS}
%\usetheme{Copenhagen}
%\usetheme{Darmstadt}
%\usetheme{Dresden}
%\usetheme{Frankfurt}
%\usetheme{Goettingen}
%\usetheme{Hannover}
%\usetheme{Ilmenau}
%\usetheme{JuanLesPins}
%\usetheme{Luebeck}
% \usetheme{Madrid}
%\usetheme{Malmoe}
%\usetheme{Marburg}
%\usetheme{Montpellier}
%\usetheme{PaloAlto}
%\usetheme{Pittsburgh}
%\usetheme{Rochester}
%\usetheme{Singapore}
% \usetheme{Szeged}
%\usetheme{Warsaw}

% As well as themes, the Beamer class has a number of color themes
% for any slide theme. Uncomment each of these in turn to see how it
% changes the colors of your current slide theme.

% \usecolortheme{albatross}
% \usecolortheme{beaver}
% \usecolortheme{beetle}
% \usecolortheme{crane}
% \usecolortheme{dolphin}
% \usecolortheme{dove}
% \usecolortheme{fly}
% \usecolortheme{lily}
% \usecolortheme{orchid}
% \usecolortheme{rose}
% \usecolortheme{seagull}
% \usecolortheme{seahorse}
\usecolortheme{whale}
% \usecolortheme{wolverine}

%\setbeamertemplate{footline} % To remove the footer line in all slides uncomment this line
%\setbeamertemplate{footline}[page number] % To replace the footer line in all slides with a simple slide count uncomment this line

%\setbeamertemplate{navigation symbols}{} % To remove the navigation symbols from the bottom of all slides uncomment this line
}

\usepackage{graphicx} % Allows including images
\usepackage{booktabs} % Allows the use of \toprule, \midrule and \bottomrule in tables
\usepackage[brazil]{babel}
\selectlanguage{brazil}
\languagepath{brazil}
\deftranslation[to=brazil]{Example}{Exemplo}
\deftranslation[to=brazil]{Theorem}{Teorema}
\usepackage[utf8]{inputenc}
\usepackage{amssymb}
\usepackage{mathtools}
\usepackage{pythonhighlight}

%----------------------------------------------------------------------------------------
%	TITLE PAGE
%----------------------------------------------------------------------------------------

\title[Locutor]{Reconhecimento de Locutor} % The short title appears at the bottom of every slide, the full title is only on the title page

\author{Ramon Duarte de Melo \& André Ribeiro Queiroz} % Your name
\institute[UFRJ] % Your institution as it will appear on the bottom of every slide, may be shorthand to save space
{
    Universidade Federal do Rio de Janeiro \\ % Your institution for the title page
    \medskip
    \textit{ramonduarte@poli.ufrj.br \& handre_queiroz@poli.ufrj.br} % Your email address
}
\date{\today} % Date, can be changed to a custom date

\begin{document}

\begin{frame} % SLIDE 1
    \titlepage % Print the title page as the first slide
\end{frame}

\begin{frame} % SLIDE 2
    \frametitle{Sumário} % Table of contents slide, comment this block out to remove it
    \tableofcontents % Throughout your presentation, if you choose to use \section{} and \subsection{} commands, these will automatically be printed on this slide as an overview of your presentation
\end{frame}

%----------------------------------------------------------------------------------------
%	PRESENTATION SLIDES
%----------------------------------------------------------------------------------------

%------------------------------------------------
\section{Introdução} % Sections can be created in order to organize your presentation into discrete blocks, all sections and subsections are automatically printed in the table of contents as an overview of the talk
%------------------------------------------------

\subsection{Definições} % A subsection can be created just before a set of slides with a common theme to further break down your presentation into chunks

\begin{frame} % SLIDE 3
    \frametitle{Glossário}
    
    \begin{block}{Recurso}
        Abstração-chave de um pedaço coeso de informação. Dentro do contexto da disciplina de \emph{Bancos de Dados}, consideraremos como um subconjunto qualquer de dados que possa ser nomeado, agrupado, processado ou referenciado: um registro, ou uma coluna, ou mesmo toda a base de dados. 
    \end{block}
    \begin{block}{Cadeado (ou \emph{lock})}
        Estrutura de dados implementada atomicamente pelo sistema operacional que oferece controle de acesso concorrente a um determinado recurso através da exclusão mútua.
    \end{block}
\end{frame}

\section{O Problema} % Sections can be created in order to organize your presentation into discrete blocks, all sections and subsections are automatically printed in the table of contents as an overview of the talk

\begin{frame} % SLIDE 3
    \frametitle{Problema do Caixeiro Viajante (TSP)}

    O problema de roteamento de veículos (VRP) é uma generalização do problema do caixeiro viajante, onde o número \(k\) de veículos representa o número máximo de circuitos admitido como resposta.

\end{frame}


\begin{frame} % SLIDE 3
    \frametitle{Problema de Roteamento de Veículos (VRP)}

    No roteamento de veículos, há um vértice especial - o \emph{armazém} - que deve ser a origem e o destino de todas as rotas, e que não será incluído na cardinalidade \(n\). Os demais vértices serão chamados de \emph{clientes}.

\end{frame}


\begin{frame} % SLIDE 3
    \frametitle{Problema de Roteamento de Veículos Capacitado (CVRP)}

    Além da limitação de distância e de veículos disponíveis, cada cliente possui uma demanda \(q_{i} \, | \, i \in \{ \, 1, \, ... \, , \, n\} \) que deve ser atendida por um dos veículos de capacidade \(c_{i} \, | \, i \in \{ \, 1, \, ... \, , \, k\} \).

\end{frame}

\section{Problemas}

\subsection{Janelas de tempo}

\begin{frame} % SLIDE 3
    \frametitle{Problema de Roteamento de Veículos Capacitado com Janelas de Tempo (CVRPTW)}

    Nesta variante, clientes só podem ser atendidos num intervalo de tempo predeterminado. Arestas passam a representar o tempo de deslocamento, em vez da distância, para evitar cálculos desnecessários de velocidade. 


\end{frame}


\begin{frame} % SLIDE 3
    \frametitle{Problema de Roteamento de Veículos Capacitado com Janelas de Tempo (CVRPTW)}

    Um veículo pode chegar ao cliente antes do intervalo, mas só poderá seguir adiante quando a janela for aberta.

    \medskip
    O problema não fica significativamente mais difícil de resolver. A dificuldade encontrada foi conferir a validade da solução, que diminui muito o \emph{throughput} do programa.

\end{frame}


\section{Limitando o CVRP}

\subsection{Dados de entrada}

\begin{frame} % SLIDE 3
    \frametitle{Grafo planar}

    Como as heurísticas testadas exigem que a \emph{inequação do triângulo} seja satisfeita, e como existem muitas instâncias deste problema já resolvidas, todas as entradas são \textbf{grafos planares}.

    \bigskip
    O grafo planar é \textbf{obrigatoriamente completo e não-direcionado} - o que elimina a necessidade de manter a lista de adjacências - e a aresta \(e(i, \, j)\) tem peso \((i_{x} - j_{x})^{2} + (i_{y} - j_{y})^{2}\).

\end{frame}


\begin{frame} % SLIDE 3
    \frametitle{Grafos solucionáveis}

    Todas as instâncias testadas têm ao menos uma solução conhecida. O objetivo é garantir que existe uma solução plausível de ser encontrada no grafo em questão.

    \bigskip
    Quando a solução ótima já é conhecida, os valores obtidos para o total de veículos utilizados e para a distância percorrida foram usadas como limites inferiores (\emph{lower bounds}).

\end{frame}

\subsection{Dados de saída}

\begin{frame} % SLIDE 3
    \frametitle{Objetivos}

    A função objetivo que determina a solução ótima do CVRP foi escolhida como a minimização de rotas, isto é, a melhor solução é a que demanda menos veículos.

    \bigskip
    Todos os veículos são considerados de mesmo custo, mesmo quando possuem capacidades diferentes (\emph{heterogêneos}). Para as heurísticas, essa dinâmica é importante porque os veículos mais pesados serão empregados primeiro.

\end{frame}

\begin{frame} % SLIDE 3
    \frametitle{Objetivos}

    Dadas duas soluções com mesmo número \(k\) de rotas empregadas, a que tiver menor distância total percorrida é considerada melhor.

    \medskip
    As duas heurísticas testadas são dependentes da escolha inicial dos vértices. Por isso, foram executadas algumas vezes com valores iniciais distintos e o resultado mostrado é o melhor encontrado.


\end{frame}

\section{Implementação}

\subsection{Algoritmos exatos}

\begin{frame}[fragile] % SLIDE 3
    \frametitle{Algoritmo 1: Força bruta}

    A força bruta é baseada no \emph{powerset} de rotas - o conjunto de todas as rotas possíveis - e no \emph{powerset} de soluções - o conjunto de todas as soluções possíveis.

    \bigskip
    \begin{python}
    def powerset(iterable, lb=1):
        "powerset([1,2,3]) --> () (1,) (2,) (3,) (1,2) 
                               (1,3) (2,3) (1,2,3)"
        s = list(iterable)
        return itertools.chain.from_iterable(
            itertools.combinations(s, r)
            for r in range(lb, len(s)+1))
    \end{python}

\end{frame}

\begin{frame}[fragile] % SLIDE 3
    \frametitle{Algoritmo 2: Força bruta com \emph{branch-and-bound}}

    Para reduzir cada \emph{powerset}, foram aplicadas algumas checagens:

    \medskip
    \begin{enumerate}
        \item limites inferiores de demanda da rota:
    \begin{block}{todo veículo em rota precisa atender ao menos um cliente}
    \begin{python}
    lower_bound = min(total_demand % max(capacities),
                      max(demands[1:]))
    \end{python}
    \end{block}

    \begin{block}{se a capacidade total for igual à demanda total, então cada veículo deve receber uma rota que consuma toda a sua capacidade}
    \begin{python}
    lower_bound = capacities[i] if (sum(capacities) ==
                                    sum(demands))
    \end{python}
    \end{block}
        
    \end{enumerate}


\end{frame}

% \begin{frame}[fragile] % SLIDE 3
%     \frametitle{Algoritmo 2: Força bruta com \emph{branch-and-bound}}

%     \begin{block}{todo veículo em rota precisa atender ao menos um cliente}
%     \begin{python}
%     lower_bound = min(total_demand % max(capacities),
%                       max(demands[1:]))
%     \end{python}
%     \end{block}

%     \begin{block}{se a capacidade total for igual à demanda total, então cada veículo deve receber uma rota que consuma toda a sua capacidade}
%     \begin{python}
%     lower_bound = capacities[i] if (sum(capacities) ==
%                                     sum(demands))
%     \end{python}
%     \end{block}
        
%     \end{enumerate}


% \end{frame}


\begin{frame}[fragile] % SLIDE 3
    \frametitle{Algoritmo 2: Força bruta com \emph{branch-and-bound}}

    \begin{enumerate}
        \setcounter{enumi}{1}
        \item limites superiores de demanda da rota:
    \begin{block}{rota precisa ter demanda compatível com a capacidade do veículo}
    \begin{python}
    upper_bound = max(capacities)
    \end{python}
    \end{block}

        \item limites inferiores de veículos assinalados:
    \begin{block}{a capacidade total precisa dar conta da demanda total}
    \begin{python}
    while sum(capacities) > sum(demands):
        capacities.pop()
    for solution in powerset(viable_routes[::-1],
                             lb=len(capacities)):
        (...)
    \end{python}
    \end{block}
        
    \end{enumerate}


\end{frame}


\begin{frame}[fragile] % SLIDE 3
    \frametitle{Algoritmo 2: Força bruta com \emph{branch-and-bound}}

    \begin{block}{se existe uma solução ótima conhecida da instância, então o número de rotas dela é um limite inferior}
    
    \end{block}

    \begin{enumerate}
        \setcounter{enumi}{3}
        \item limites superiores de veículos assinalados:

    \begin{block}{se existe uma solução heurística viável, então a solução ótima tem que ter um número de rotas obrigatoriamente menor ou igual}
    \begin{python}
    upper_bound = heuristics_bounds()["upper"]
    \end{python}
    \end{block}
        
    \end{enumerate}

\end{frame}


\begin{frame}[fragile] % SLIDE 3
    \frametitle{Algoritmo 2: Força bruta com \emph{branch-and-bound}}

    O \emph{powerset} gera um conjunto ordenado de rotas. Portanto, se uma rota \(r_{i}\) for incompatível com outra \(r_{j > i}\), é abandonar o fio de execução em favor da rota \(r_{j+1}\). 

    \bigskip

    \begin{enumerate}
        \setcounter{enumi}{4}
        \item checagem de soluções:

    \begin{block}{se duas rotas possuem um vértice em comum, então as soluções que incluem ambas são inviáveis:}
    \begin{python}
    if (len(set(i for x in solution for i in x[0]))
        != len([i for x in solution for i in x[0])]):
        continue
    \end{python}
    \end{block}
        
        
    \end{enumerate}

\end{frame}


\begin{frame}[fragile] % SLIDE 3
    \frametitle{Algoritmo 2: Força bruta com \emph{branch-and-bound}}

    \begin{block}{se uma rota faz a solução ter distância maior que a encontrada por uma heurística, então ela não leverá a uma solução ótima:}
    \begin{python}
    if this_solution_distance > distance_upper_bound:
        continue
    \end{python}
    \end{block}

    \begin{block}{se a rota introduz uma solução que é pior que a melhor solução já encontrada pelo algoritmo, então ela não levará ao ótimo:}
    \begin{python}
    total_weight = sum(
        [G.get_edge_data(x[0][i], x[0][i+1])["weight"]
         for x in solution for i in range(len(x[0])-1)])
    if total_weight > best_route[1]:
        continue
    \end{python}
    \end{block}

        
\end{frame}



\begin{frame} % SLIDE 3
    \frametitle{Objetivos}

    Dadas duas soluções com mesmo número \(k\) de rotas empregadas, a que tiver menor distância total percorrida é considerada melhor.

    \medskip
    As duas heurísticas testadas são dependentes da escolha inicial dos vértices. Por isso, foram executadas algumas vezes com valores iniciais distintos e o resultado mostrado é o melhor encontrado.


\end{frame}

\section{Referências bibliográficas}

\begin{frame} % SLIDE 53
    \frametitle{Referências bibliográficas}
    \footnotesize{
    \begin{thebibliography}{99} % Beamer does not support BibTeX so references must be inserted manually as below
    % \bibitem[Smith, 2012]{p1} John Smith (2012)
        \bibitem[Silberschatz, 2010]{p1}
        Silberschatz, A., Korth, H. F., \& Sudarshan, S. (2010). Database system concepts (Vol. 4). New York: McGraw-Hill.

        \bibitem[Rawat, 2017]{p2}
        Rawat, U. (2017). Implementation of Locking in DBMS. Acessado a \date{25/11/2018} em https://www.geeksforgeeks.org/implementation-of-locking-in-dbms/.

        \bibitem[Porfirio, 2013]{p3}
        Porfirio, Alice \& Pellegrini, Alessandro \& Di Sanzo, Pierangelo \& Quaglia, Francesco. (2013). Transparent Support for Partial Rollback in Software Transactional Memories. 8097. 583-594. 10.1007/978-3-642-40047-6\_59. 

        \bibitem[Poddar, 2013]{p4}
        Poddar, Saumendra. (2003). SQL Server Transactions and Error Handling. Acessado a \date{25/11/2018} em https://www.codeproject.com/Articles/4451/SQL-Server-Transactions-and-Error-Handling.
    \end{thebibliography}
    }
\end{frame}


\begin{frame} % SLIDE 53
    \frametitle{Referências bibliográficas}
    \footnotesize{
    \begin{thebibliography}{99} % Beamer does not support BibTeX so references must be inserted manually as below

    \bibitem[Singhal, 2018]{p5}
        Singhal, Akshay. (2018). Cascading Schedule | Cascading Rollback | Cascadeless Schedule. Acessado a \date{25/11/2018} em https://www.gatevidyalay.com/cascading-schedule-cascading-rollback-cascadeless-schedule/.

        \bibitem[Pandey, 2018]{p6}
        Pandey, Anand. (2018). Transactions and Concurrency Control. Acessado a \date{25/11/2018} em https://gradeup.co/transactions-and-concurrency-control-i-4c5d9b27-c5a7-11e5-bcc4-bc86a005f7ba.

	    \bibitem[Difference between, 2018]{p7}
        Difference between. (2018). Difference between Deadlock and Starvation. Acessado a \date{25/11/2018} em http://www.differencebetween.info/difference-between-deadlock-and-starvation.

    \end{thebibliography}
    }
\end{frame}

%------------------------------------------------

\end{document} 